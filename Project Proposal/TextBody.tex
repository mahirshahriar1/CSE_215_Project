\documentclass{article}
\title{Sections and Chapters}
\author{Overleaf}
\date{\today}
\usepackage{hyperref}

\hypersetup{
    colorlinks=true,
    linkcolor=dark blue,
    filecolor=magenta,      
    urlcolor=blue,
    pdftitle={Overleaf Example},
    pdfpagemode=FullScreen,
 
    }
    
\urlstyle{same}

\begin{document}

\addcontentsline{toc}{section}{Front Page}
\section*{Front Page}
\pagebreak

\addcontentsline{toc}{section}{Table Of Contents}
{\Large \bf\tableofcontents}

\pagebreak

\addcontentsline{toc}{section}{Introduction}
\section*{\LARGE Introduction}
\large With the increase in population, the need of larger supermarkets are inevitable. 
However, humans are bound to make errors while keeping tabs on such huge inventory. This is where this software will come handy. A software where the admin will be able to add or remove products and  will also be able to assign the prices. The software will have staff access where they will be able to sell a product and the inventory will automatically be updated upon sale.


\addcontentsline{toc}{section}{Objective}
\section*{\LARGE Objective}
\large 
\begin{itemize}
    \item \large To Make Things EASY!
    \item \large To easily add or remove products from inventory. 
    \item \large To easily add or update price of a product.
    \item \large Staffs can log into their account and sell products.
    \item \large Automatically sums up the product's prices while selling.
    \item \large Will automatically update the inventory once products are sold.
    
\end{itemize}

\addcontentsline{toc}{section}{Target Customers}
\section*{\LARGE Target Customers}
\large 
\begin{itemize}
    \item \large Any Supermarket Owners can use this software without any further changes. 
    \item \large Other Product Selling Companies can also adopt this software in their management with little changes made in the interface.
\end{itemize}

\addcontentsline{toc}{section}{Value Proposition}
\section*{\LARGE Value Proposition}
\large The software will reduce time to keep track of the inventory and manually updating prices or number of products left. The owner will always have sound knowledge of what is in their store and can take necessary decisions upon seeing the data. Staffs will also have easy time to sell the products by just selecting and adding the quantity of being sold. The software will automatically sum up the total.

\pagebreak

\addcontentsline{toc}{section}{Software Features And Descriptions}
\section*{\LARGE Software Features And Descriptions}
The software will open with a Login option at first. If the admin wants to access it, they need to select ""Admin Access" and give the password. If any staff wants to access, they need to select "Staff Access" and enter a password. \\

\textbf {\Large Admin's screen will have:}
\begin{itemize}
    \item \large Add a product.
    \item \large Remove a product.
    \item \large Update a product's information.
    \item \large Delete a product.
    \item \large Check Profit Made.  \\
\end{itemize} 

\textbf {\Large Staff's screen will have:}
\begin{itemize}
    \item \large Sell a product.
    \begin{itemize}
    \item \large Enter Product ID.
    \item \large Enter Quantity.
    \end{itemize}
    \item \large Check Inventory List.  \\
\end{itemize} 

\addcontentsline{toc}{section}{Tools and Resources}
\section*{\LARGE Tools and Resources}
\Large -Java\\
\Large -Java Swing for GUI
 
 \pagebreak
\addcontentsline{toc}{section}{Challenge}
\section*{\LARGE Challenge}
\large One of the big obstacle is design. The software needs to be designed in such a way that any person with minimal effort can use and utilize it successfully. Otherwise, we will fail to meet our main objective, which is to make things easier.\\\\\\

\centering \underline{\textbf{\Large THE END}}\\

\\\LARGE\underline{\href{https://github.com/mahirshahriar1/CSE_215_Project}{\\Click here to access Github} }


\end{document}t
